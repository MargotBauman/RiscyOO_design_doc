\subsection{MMIO Hanlder}\label{sec:mmio-core}

The MMIO handler delivers the following MMIO accesses made inside the processor core to the uncore platform:
\begin{itemize}
    \item boot-rom accesses to fetch instructions, and
    \item MMIO data accesses made by memory instructions in LSQ.
\end{itemize}
The module also processes the following requests from the uncore platform:
\begin{itemize}
    \item posting timer interrupt (the \code{MTIP} bit),
    \item reading and writing the software interrupt bit (\code{MSIP}), and
    \item updating the copy of the machine time in CSR register file.
\end{itemize}

\subsubsection{Interface}

\begin{figure}
\begin{lstlisting}[caption={}]
interface MMIOCoreToPlatform;
  interface FifoDeq#(MMIOCRq) cRq;
  interface FifoEnq#(MMIOPRs) pRs;
  interface FifoEnq#(MMIOPRq) pRq;
  interface FifoDeq#(MMIOCRs) cRs;
  method Action setTime(Data t);
endinterface
interface MMIOCore;
  method Bool isMMIOAddr(Addr addr);
  method Action dataReq(MMIOCRq r);
  method MMIODataPRs dataRespVal;
  method Action dataRespDeq;
  method Action setHtifAddrs(Addr toHost, Addr fromHost);
  method Bool hasPendingPRq;
  interface MMIOCoreToPlatform toP;
endinterface
interface MMIOInstToCore;
  interface FifoDeq#(Tuple2#(Addr, SupWaySel)) instReq;
  interface FifoEnq#(Vector#(SupSize, Maybe#(Instruction))) instResp;
  method Action setHtifAddrs(Addr toHost, Addr fromHost);
endinterface
interface MMIOCoreInput;
  interface MMIOInstToCore fetch;
  method Bit#(1) getMSIP;
  method Action setMSIP(Bit#(1) v);
  method Action setMTIP(Bit#(1) v);
  method Action setTime(Data t);
  method Bool noInflightCSRInstOrInterrupt;
endinterface
module mkMMIOCore#(MMIOCoreInput inIfc)(MMIOCore);
  // implementation
endmodule
\end{lstlisting}
\caption{Interface of MMIO handler}\label{fig:mmio-core-ifc}
\end{figure}

Figure~\ref{fig:mmio-core-ifc} shows the interface of the MMIO handler module.
Module interface \code{MMIOCore} contains the following fields:
\begin{itemize}
    \item Method \code{isMMIOAddr}: returns true if physical address \code{addr} is a MMIO address.
    The \code{tohost} and \code{fromhost} addresses used in the BBL (Berkeley boot loader) are also considered as MMIO addresses.
    \item Method \code{dataReq}: sends a MMIO requests to uncore platform.
    \item Methods \code{dataRespVal} and \code{dataRespDeq}: gets the MMIO response from uncore platform.
    \item Method \code{setHtifAddrs}: tells the MMIO handler what are the \code{tohost} and \code{fromhost} addresses.
    \item Method \code{hasPendingPRq}: returns true if the handler has received a request from the uncore platform.
    The rename stage calls this method to stop renaming and let the MMIO handler to complete the request from uncore platform.
    This can avoid conflicting accesses on CSRs, because rename stage checks interrupts while the uncore-platform request may post interrupts.
    \item Subinterface \code{toP}: contains FIFO interfaces connected to the uncore platform.
    It also contains a \code{setTime} methods to let the uncore platform to update the copy of machine time in CSR register file.
\end{itemize}
Interface \code{MMIOCoreInput} is the argument passed  to the module.
It contains the following fields:
\begin{itemize}
    \item Subinterface \code{fetch}: contains FIFO interfaces of the fetch pipeline to access boot rom in the uncore platform.
    \item Methods \code{getMSIP}, \code{setMSIP}, \code{setMTIP}, and \code{setTime}: are from CSR register file to update the CSRs.
    \item Method \code{noInflightCSRInstOrInterrupt}: returns true if the ROB  does not contain any \inst{CSRRW} instructions or interrupt instructions.
    The MMIO handler processes a request from uncore platform only when this method returns true to avoid conflicting accesses on CSRs.
    (Exceptions do not touch the interrupt bits, so it is ok for ROB to contain instructions with exceptions.)
    This may be over conservative, because RISC-V ISA seems to guarantee that software cannot modify \code{MSIP} and \code{MTIP} bits.
\end{itemize}

\subsubsection{Implementation}
The module allows only one outstanding a MMIO request from the core, because MMIO accesses are believed to be rare.

When the module receives a request from the uncore platform to read or write the interrupt pending bits, it takes actions only when the \code{noInflightCSRInstOrInterrupt} method of the module argument interface returns true.
After taking the action, it sends a response to the uncore platform.

\subsubsection{Future Improvement}
As mentioned earlier, checking method \code{noInflightCSRInstOrInterrupt} before handling requests from uncore platform may be too conservative.
We should consider removing this check in the future.

\subsubsection{Source Code}
See module \code{mkMMIOCore} in \code{//procs/lib/MMIOCore.bsv}.
