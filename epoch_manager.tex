\subsection{Epoch Manager}

The epoch manager manages the epoch that is used to kill wrong-path instructions at the rename stage.
Each time a redirection happens, the epoch is incremented by one, so the epoch value will change as follows (\code{NumEpochs} is the number of unique epoch values):
\begin{center}
    0 $\rightarrow$ 1 $\rightarrow$ $\cdots$ $\rightarrow$ \code{NumEpochs}$-1$ $\rightarrow$ 0 $\rightarrow$ $\cdots$
\end{center}
The module also tracks the range of epoch values being used by instructions in the fetch pipeline.
The epoch cannot be incremented if the next epoch value falls within the range.
The range is updated by looking at the epochs of instructions coming out of the fetch pipeline.

\subsubsection{Interface}
Figure~\ref{fig:epoch-ifc} shows the interface of the epoch manager.
Now we explain each interface method:
\begin{itemize}
    \item Subinterface \code{checkEpoch}: provides a vector of methods for the rename stage to check if an instruction is at the wrong path.
    The input argument \code{e} is the epoch carried by the instruction.
    The method compares \code{e} with the internal epoch of the module, and returns false if the instruction is at the wrong path and should be killed.
    
    \item Subinterface \code{updatePrevEpoch}: provides a vector of methods to update the latest epoch values seen from the fetch pipeline.
    Every instruction that arrives at the rename stage (including wrong-path instructions) will call this method to help the module recycle unused epoch values.
    
    \item Method \code{incrementEpoch}: is called when PC needs to be redirect.
    This method increased the internal epoch of the module.
    The guard will be false if the next value of the epoch is not yet recycled (i.e., still being used by instructions in the fetch pipeline).
\end{itemize}

\begin{figure}
\begin{lstlisting}[caption={}]
interface EM_checkEpoch;
  method Bool check(Epoch e);
endinterface
interface EM_updatePrevEpoch;
  method Action update(Epoch e);
endinterface
interface EpochManager;
  interface Vector#(SupSize, EM_checkEpoch) checkEpoch;
  interface Vector#(SupSize, EM_updatePrevEpoch) updatePrevEpoch;
  method Epoch getEpoch;
  method Action incrementEpoch;
endinterface
\end{lstlisting}
\caption{Interface of epoch manager}\label{fig:epoch-ifc}
\end{figure}

\noindent\textbf{Conflict Matrix:}
The conflict matrix of the interface methods is:
\begin{enumerate}
    \item \code{checkEpoch} $<$ \code{incrementEpoch}
    \item \code{updatePrevEpoch} CF \{\code{checkEpoch}, \code{incrementEpoch}\}
\end{enumerate}
We make method \code{updatePrevEpoch} conflict free with other methods, because the recycle of unused epoch values does not need to happen immediately.

\subsubsection{Implementation}
The module has two internal registers \code{curr\_epoch} and \code{prev\_checked\_epoch}.
\code{curr\_epoch} is the current epoch value which is used to kill wrong-path instructions at the rename stage.
\code{prev\_checked\_epoch} is the last seen epoch value from the fetch pipeline.
Thus, the range of epoch values being used is: \code{prev\_checked\_epoch}, \code{prev\_checked\_epoch}$+1$, $\ldots$, \code{curr\_epoch}.

Methods \code{checkEpoch} and \code{incrementEpoch} access directly the registers.
For subinterface \code{updatePrevEpoch}, we first capture its arguments in wires, and then update the \code{prev\_checked\_epoch} register.
Wires are ok because \code{updatePrevEpoch} is conflict free with other methods.

\subsubsection{Source Code}
See module \code{mkEpochManager} in \code{//procs/lib/EpochManager.bsv}.        
