\subsection{Aggressive (Optimistic) Scoreboard}\label{sec:sbaggr}

The aggressive scoreboard holds an optimistic version of the presence bits of all the physical registers.
Instructions in the execution pipeline can optimistically set the presence bits in this scoreboard even when the instruction has not yet finished execution (but it may be close to finish).
The renaming stage will check the presence bits from this scoreboard for the source registers of the renaming instruction, and unset the present bit in this scoreboard for the destination register of the renaming instruction.

The interface and module presented here are slightly different from what is the source code, because the module in the source code is used for not only this aggressive scoreboard but also another conservative scoreboard (Section~\ref{sec:prf+sbcons}).

\subsubsection{Interface}
Figure~\ref{fig:aggr-sb-ifc} shows the interface of this module.
Now we explain each interface method: 
\begin{itemize}
    \item Subinterface \emph{setReady}: provides a vector of methods for instructions in execution pipelines to set the presence bits of their physical destination registers optimistically.
    \item Subinterface \emph{eagerLookup}: provides a vector of methods for instructions at the renaming stage to check the presence bits of their physical source registers.
    \item Subinterface \emph{setBusy}: provides a vector of methods for instructions at the renaming stage to unset the presence bits of their physical destination registers.
\end{itemize}

\begin{figure}[!htb]
\begin{lstlisting}[caption={}]
// PhyRegs is a struct containing all the physical regs of an instruction
// RegsReady is a struct containing the presence bits for all the physical regs of an instruction
// PhyRIndex is the index of a physical reg
// SupSize is the superscalar size
interface SbLookup;
  method RegsReady get(PhyRegs r);
endinterface
interface SbSetBusy;
  method Action set(Maybe#(PhyRIndx) dst);
endinterface
interface ScoreboardAggr#(numeric type setReadyNum);
  interface Vector#(setReadyNum, Put#(PhyRIndx)) setReady;
  interface Vector#(SupSize, SbLookup) eagerLookup;
  interface Vector#(SupSize, SbSetBusy) setBusy;
endinterface
module mkScoreboardAggr(ScoreboardAggr#(setReadyNum));
  // module implementation
endmodule
\end{lstlisting}
\caption{Interface of the aggressive scoreboard}\label{fig:aggr-sb-ifc}
\end{figure}

\noindent\textbf{Conflict Matrix:}
The conflict matrix of the interface methods is:
\begin{center}
    setReady[0] $<$ $\cdots$ $<$ setReady[setReadyNum-1] $<$ eagerLookup[0] $<$ setBusy[0] $<$ eagerLookup[1] $<$ setBusy[1] $<$ $\cdots$ $<$ eagerLookup[SupSize-1] $<$ setBusy[SupSize-1].
\end{center}
We put all methods in a total order, though the conflict matrix does not need to transitive.
We order setReady $<$ eagerLookup to match the rule ordering between instruction-execution rules (which call setReady) and the renaming rule (which calls eagerLookup).
The rule ordering is because the execution rules are in later stages of the pipeline than renaming.
Thus, eagerLookup will observe the effects of all the calls to setReady in the same cycle, and this is why we call it ``eager''.
Methods eagerLookup[$i$] and setBusy[$i$] are used by the $i^{th}$ renamed instruction at the renaming stage in each cycle.
Since the $i^{th}$ instruction should see the renaming effects of all previous instructions, we order setBusy[$0\ldots i-1$] $<$ eagerLookup[$i$].

\subsubsection{Implementation}
The implementation of the module does not contain any internal rules.
It just uses a vector of EHRs to hold the presence bits, one EHR for each bit.
The interface methods access the EHRs using the appropriate port according to the conflict matrix.

\subsubsection{Source Code}
See the followings:
\begin{itemize}
    \item module \texttt{mkScoreboardAggr} in \texttt{//procs/RV64G\_OOO/ScoreboardSynth.bsv}, and
    \item module \texttt{mkRenamingScoreboard} in file \texttt{//procs/lib/Scoreboard.bsv}.
\end{itemize}
  