\subsection{FPU}~\label{sec:fpu}

The FPU module contains the following floating-point execution units:
\begin{itemize}
    \item Simple unit: performs simple operations, e.g., converting between integer and floating point.
    \item FMA (fused-multiply-add) unit: performs add, multiply and FMA.
    \item Divide unit: performs division.
    \item Square-root unit: calculates the square root.
\end{itemize}
Every floating point instruction will use one of the execution units.
Since different execution units can have different latency and throughput, FPU will respond out of order.
FPU also keeps the speculation mask for each instruction inside, so wrong-path instructions will be dropped inside FPU and will not be outputted.

\subsubsection{Interface}

\begin{figure}
\begin{lstlisting}[caption={}]
interface FpuExec;
  method Action exec(FpuInst fpu_inst, Data rVal1, Data rVal2, Data rVal3,
  Maybe#(PhyDst) dst, InstTag tag, SpecBits specBits);
  method ActionValue#(FpuResp) simpleResp;
  method ActionValue#(FpuResp) fmaResp;
  method ActionValue#(FpuResp) divResp;
  method ActionValue#(FpuResp) sqrtResp;
  interface SpeculationUpdate specUpdate;
endinterface
module mkFpuExecPipeline(FpuExec);
  // implementation
endmodule
\end{lstlisting}
\caption{Interface of FPU}\label{fig:fpu-ifc}
\end{figure}

Figure~\ref{fig:fpu-ifc} shows the interface of FPU:
\begin{itemize}
    \item Method \code{exec}: is called to send a new floating point instruction \code{fpu\_inst} into the FPU.
    This method will dispatch the instruction to the appropriate execution unit.
    Notice that argument \code{specBits} is used to kill wrong-path instructions inside FPU.
    \item Methods \code{simpleResp}, \code{fmaResp}, \code{divResp}, \code{sqrtResp}: return the responses from the simple, FMA, divide, and square-root units, respectively.
    Wrong-path instructions will not be returned.
    \item Subinterface \code{specUpdate}: manipulates speculative states (Section~\ref{sec:specupdate}).
\end{itemize}

\noindent\textbf{Conflict Matrix:}
\begin{itemize}
    \item \{\code{simpleResp}, \code{fmaResp}, \code{divResp}, \code{sqrtResp}\} $<$ \code{exec}.
    \item \code{simpleResp}, \code{fmaResp}, \code{divResp}, \code{sqrtResp} are conflict-free with each other.
\end{itemize}


\subsubsection{Implementation}
The simple unit is just a speculation FIFO (Section~\ref{sec:specfifo}).
For an instruction that goes into the simple unit, the \code{exec} method directly computes the result and enters the result into the FIFO.

The FMA, divide and square-root units are implemented in the same way.
For each unit, the real computation pipeline is a Xilinx IP core.
There is a speculative poisoned FIFO (Section~\ref{sec:specpoisonfifo}) accompanying the pipeline.
The FIFO contains all the in-flight instructions in the pipeline.
When mis-speculation happens, wrong-path instructions will be poisoned in the FIFO.
And when the wrong-path instruction is dequeued from the pipeline and the FIFO, it will be dropped.

\subsubsection{Source Code}
See the followings:
\begin{itemize}
    \item module \code{mkFpuExecPipeline} in \code{//procs/lib/Fpu.bsv}, and
    \item all modules in \code{//fpgautils/lib/XilinxFpu.bsv}.
\end{itemize}
