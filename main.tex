\documentclass[12pt]{article}
% my packages
\usepackage{fullpage}
\usepackage{amsmath}
\usepackage{amsthm}
\usepackage{amssymb}
\usepackage{mathtools}
\usepackage{verbatim}
\usepackage{color}
\usepackage{boxedminipage}
\usepackage{slashbox}
\usepackage{enumitem}
\usepackage[caption=false]{subfig}
\usepackage{url}
\usepackage{listings}
\usepackage{xcolor}
\usepackage{hyperref}

\definecolor{commentcolor}{rgb}{0,0.6,0}
\definecolor{keywordcolor}{rgb}{0,0,0.8}
\definecolor{numbercolor}{rgb}{0.5,0.5,0.5}
\definecolor{stringcolor}{rgb}{0.58,0,0.82}
\lstset{%
    language=Verilog,                        % closest to BSV
    backgroundcolor=\color{white},           % choose the background color; you must add \usepackage{color} or \usepackage{xcolor}
    basicstyle=\ttfamily\bfseries,              % the size of the fonts that are used for the code
    belowskip=0.5\baselineskip,            % from: http://tex.stackexchange.com/questions/118730/avoid-empty-vert-space-after-lstlisting
    breakatwhitespace=false,                 % sets if automatic breaks should only happen at whitespace
    breaklines=true,                         % sets automatic line breaking
    captionpos=b,                            % sets the caption-position to bottom
    commentstyle=\color{commentcolor},       % comment style
    deletekeywords={...},                    % if you want to delete keywords from the given language
    escapeinside={\%*}{*)},                  % if you want to add LaTeX within your code
    extendedchars=true,                      % lets you use non-ASCII characters; for 8-bits encodings only, does not work with UTF-8
    frame=single,                            % adds a frame around the code
    %frame=none,                              % doesn't add a frame around the code
    keepspaces=true,                         % keeps spaces in text, useful for keeping indentation of code (possibly needs columns=flexible)
    %columns=fixed,                           %
    columns=flexible,                        %
    keywordstyle=\color{keywordcolor},       % keyword style
    morekeywords={%
        call,%Called method
        def,%Defined method
        inverted,%Inverted interface
        type,typedef,valueOf,                                    % Type related keywords
        method,endmethod,action,endaction,                  % Methods and actions
        Action,ActionValue,interface,endinterface,          % Interfaces
        Vector,replicate,replicateM,                        % Vector stuff
        Bit,Int,UIng,Reg,Integer,let,tagged,union,struct,   % Basic types
        TAdd,TMul,TDiv,                                     % Type operations
        rule,endrule,return,                                % Rules keywords
        pack,unpack,zeroExtend,signExtend,                  % Common bit functions
        case,matches,endcase                                % Case statements
        synthesize,True,False,Empty,*,...},                 % Etc.
    numbers=left,                            % where to put the line-numbers; possible values are (none, left, right)
    numbersep=5pt,                           % how far the line-numbers are from the code
    numberstyle=\tiny\color{numbercolor},    % the style that is used for the line-numbers
    rulecolor=\color{black},                 % if not set, the frame-color may be changed on line-breaks within not-black text (e.g. comments (green here))
    showspaces=false,                        % show spaces everywhere adding particular underscores; it overrides 'showstringspaces'
    showstringspaces=false,                  % underline spaces within strings only
    showtabs=false,                          % show tabs within strings adding particular underscores
    stepnumber=1,                            % the step between two line-numbers. If it's 1, each line will be numbered
    stringstyle=\color{stringcolor},         % string literal style
    tabsize=2,                               % sets default tabsize to 2 spaces
    title=\lstname                           % show the filename of files included with \lstinputlisting; also try caption instead of title
}

\newcommand{\mycomment}[1]{\emph{\textcolor{red}{[#1]}}}

\begin{document}

\title{RiscyOO Design Document}
\author{Sizhuo Zhang \\ szzhang@csail.mit.edu \\ MIT CSAIL}
\date{}
\maketitle

\section{Overview}


\section{Processor Core}

Some common things:
\begin{itemize}
    \item Rules in later pipeline stages are ordered before rules in earlier pipeline stages.
\end{itemize}

\subsection{Aggressive (Optimistic) Scoreboard}\label{sec:sbaggr}

The aggressive scoreboard holds an optimistic version of the presence bits of all the physical registers.
Instructions in the execution pipeline can optimistically set the presence bits in this scoreboard even when the instruction has not yet finished execution (but it may be close to finish).
The renaming stage will check the presence bits from this scoreboard for the source registers of the renaming instruction, and unset the present bit in this scoreboard for the destination register of the renaming instruction.

The interface and module presented here are slightly different from what is the source code, because the module in the source code is used for not only this aggressive scoreboard but also another conservative scoreboard (Section~\ref{sec:prf+sbcons}).

\subsubsection{Interface}
Figure~\ref{fig:aggr-sb-ifc} shows the interface of this module.
Now we explain each interface method: 
\begin{itemize}
    \item Subinterface \emph{setReady}: provides a vector of methods for instructions in execution pipelines to set the presence bits of their physical destination registers optimistically.
    \item Subinterface \emph{eagerLookup}: provides a vector of methods for instructions at the renaming stage to check the presence bits of their physical source registers.
    \item Subinterface \emph{setBusy}: provides a vector of methods for instructions at the renaming stage to unset the presence bits of their physical destination registers.
\end{itemize}

\begin{figure}
\begin{lstlisting}[caption={}]
// PhyRegs is a struct containing all the physical regs of an instruction
// RegsReady is a struct containing the presence bits for all the physical regs of an instruction
// PhyRIndex is the index of a physical reg
// SupSize is the superscalar size
interface SbLookup;
  method RegsReady get(PhyRegs r);
endinterface
interface SbSetBusy;
  method Action set(Maybe#(PhyRIndx) dst);
endinterface
interface ScoreboardAggr#(numeric type setReadyNum);
  interface Vector#(setReadyNum, Put#(PhyRIndx)) setReady;
  interface Vector#(SupSize, SbLookup) eagerLookup;
  interface Vector#(SupSize, SbSetBusy) setBusy;
endinterface
module mkScoreboardAggr(ScoreboardAggr#(setReadyNum));
  // module implementation
endmodule
\end{lstlisting}
\caption{Interface of the aggressive scoreboard}\label{fig:aggr-sb-ifc}
\end{figure}

\noindent\textbf{Conflict Matrix:}
The conflict matrix of the interface methods is:
\begin{center}
    setReady[0] $<$ $\cdots$ $<$ setReady[setReadyNum-1] $<$ eagerLookup[0] $<$ setBusy[0] $<$ eagerLookup[1] $<$ setBusy[1] $<$ $\cdots$ $<$ eagerLookup[SupSize-1] $<$ setBusy[SupSize-1].
\end{center}
We put all methods in a total order, though the conflict matrix does not need to transitive.
We order setReady $<$ eagerLookup to match the rule ordering between instruction-execution rules (which call setReady) and the renaming rule (which calls eagerLookup).
The rule ordering is because the execution rules are in later stages of the pipeline than renaming.
Thus, eagerLookup will observe the effects of all the calls to setReady in the same cycle, and this is why we call it ``eager''.
Methods eagerLookup[$i$] and setBusy[$i$] are used by the $i^{th}$ renamed instruction at the renaming stage in each cycle.
Since the $i^{th}$ instruction should see the renaming effects of all previous instructions, we order setBusy[$0\ldots i-1$] $<$ eagerLookup[$i$].

\subsubsection{Implementation}
The implementation of the module does not contain any internal rules.
It just uses a vector of EHRs to hold the presence bits, one EHR for each bit.
The interface methods access the EHRs using the appropriate port according to the conflict matrix.

\subsubsection{Source Code}
See the followings:
\begin{itemize}
    \item module \texttt{mkScoreboardAggr} in \texttt{//procs/RV64G\_OOO/ScoreboardSynth.bsv}, and
    \item module \texttt{mkRenamingScoreboard} in file \texttt{//procs/lib/Scoreboard.bsv}.
\end{itemize}
  

\subsection{Physical Register File and Conservative Scoreboard}\label{sec:prf+sbcons}

The physical register file contains the value for each physical register, and the conservative scoreboard contains the presence bit for each physical-register value.
Instructions at the renaming stage will unset the presence bits of the destination physical registers.
Instructions that just finish execution write data into the physical register file and set the presence bits simultaneously.
Instructions in the execution pipeline read both the physical register file and the scoreboard to get the source operand and whether the operand data is present or not.
The scoreboard is called conservative because each presence bit is set only when the data is written to the corresponding physical register.

Although the physical register file and the conservative scoreboard are currently implemented as two separate modules, we describe them as one module here because they should be accessed together.

\subsubsection{Interface}
Figure~\ref{fig:prf-sb-ifc} shows the interface of the fused module of the physical register file and the conservative scoreboard.
It should be noted that the interface shown here is different from what is in the source code, which splits this module as two separate modules.

\begin{figure}[!htb]
\begin{lstlisting}[caption={}]
// PhyRIndex is the index of a physical reg
// SupSize is the superscalar size
interface RFileWr;
  method Action wr(PhyRIndx rindx, Data data);
endinterface
interface RFileRd;
  method Maybe#(Data) rd1(PhyRIndx rindx);
  method Maybe#(Data) rd2(PhyRIndx rindx);
  method Maybe#(Data) rd3(PhyRIndx rindx);
endinterface
interface SbSetBusy;
  method Action set(Maybe#(PhyRIndx) dst);
endinterface
interface RFileSbCons#(numeric type wrNum, numeric type rdNum);
  interface Vector#(wrNum, RFileWr) write;
  interface Vector#(rdNum, RFileRd) read;
  interface Vector#(SupSize, SbSetBusy) setBusy;
endinterface
module mkRFileSbCons(RFileSbCons#(wrNum, rdNum));
  // module implementation
endmodule
\end{lstlisting}
\caption{Interface of physical register file and conservative scoreboard}\label{fig:prf-sb-ifc}
\end{figure}

Now we explain each interface method:
\begin{itemize}
    \item Subinterface \emph{write}: provides a vector of methods for instructions that just finish execution to write the data into the physcial register file and set the presence bit.
    \item Subinterface \emph{read}: provides a vector of methods for instructions at register-read stage to read both the data and presence bit of the source register.
    In case the presence bit is unset, the method returns \texttt{Invalid}.
    \item Subinterface \emph{setBusy}: provides a vector of methods for instructions at the renaming stage to unset the presence bits of their physical destination registers.
    this method will be called together with the one in the aggressive scoreboard.
\end{itemize}

The conflict matrix of the interface methods is:
\begin{itemize}
    \item read CF \{write, setBusy\}
    \item write[0] $<$ write[1] $<$ $\cdots$ $<$ write[wrNum-1] $<$ setBusy[0] $<$ setBusy[1] $<$ $\cdots$ $<$ setBusy[wrNum-1]
\end{itemize}
We set all the read methods conflict free with all the write methods, because reads do not need to reflect the effects of writes immediately.
In case an old instruction is writing a register which is being read by a younger instruction as a source operand, it is ok for the read method to return \texttt{Invalid} even if the write method has been called.
This simply delays the execution of the younger instruction.
In our implementation, the read method will return \texttt{Invalid} if a write for the same physical register is being performed in the same cycle.
This can cut off the combinational path.

Forcing read methods to be ordered before write methods can cause rule-scheduling problems, because rules that read the register file are in earlier pipeline stages than rules that write register file.
That is, the method ordering will not agree with the (desired) top-level rule ordering.

We also set all the read methods conflict free with all the setBusy methods.
This is because the renaming algorithm guarantees that a physical register that is being used as a source operand of a (currently correct-path) instruction cannot be a free register for renaming destination architectural registers.
However, \emph{it is still desirable in the furture implementation to set all the read methods $<$ all the setBusy methods}, because reads happen in later pipeline stages than setBusy does.

We order all the write methods before all the setBusy methods, because write methods are called in later stages in the pipeline than setBusy methods.

\subsubsection{Implementation}
The implementation uses a vector of EHRs to store the register data, and another vector of EHRs to store the presence bits.
The write methods and set busy methods directly access the EHRs using appropriate ports according to the conflict matrix of the interface.

The read methods do not directly access the EHRs.
We use a vector of wires to retrieve value of port 0 of each EHR, and the read methods read from these wires.
This prevents any combinational path from write to read.
This is a valid implementation because the read methods are conflict free with the write and setBusy methods, and the read methods can safely ignore the effects caused by the write or setBusy methods.

\subsubsection{Source Code}
See the followings:
\begin{itemize}
    \item module \texttt{mkRFileSynth} in file \texttt{//procs/RV64G\_OOO/RFileSynth.bsv},
    \item module \texttt{mkRFile} in file \texttt{//procs/lib/PhysRFile.bsv},
    \item module \texttt{mkScoreboardCons} in file \texttt{//procs/RV64G\_OOO/ScoreboardSynth.bsv}, and
    \item module \texttt{mkRenamingScoreboard} in file \texttt{//procs/lib/Scoreboard.bsv}.
\end{itemize}

\subsubsection{Future Improvement}
There are two improvements to make in a future implementation:
\begin{enumerate}
    \item Merge the physical register file and the conservative scoreboard into one module to ensure that their methods are called together.
    \item Order the read methods before the setBusy methods, so we don't need to rely on the high-level invariant of the renaming algorithm to ensure correctness.
\end{enumerate}

\section{Uncore}

\end{document}