\documentclass[12pt]{article}
% my packages
\usepackage{fullpage}
\usepackage{amsmath}
\usepackage{amsthm}
\usepackage{amssymb}
\usepackage{mathtools}
\usepackage{verbatim}
\usepackage{color}
\usepackage{boxedminipage}
\usepackage{slashbox}
\usepackage{enumitem}
\usepackage[caption=false]{subfig}
\usepackage{url}
\usepackage{listings}
\usepackage{xcolor}
\usepackage{hyperref}

\definecolor{commentcolor}{rgb}{0,0.6,0}
\definecolor{keywordcolor}{rgb}{0,0,0.8}
\definecolor{numbercolor}{rgb}{0.5,0.5,0.5}
\definecolor{stringcolor}{rgb}{0.58,0,0.82}
\lstset{%
    language=Verilog,                        % closest to BSV
    backgroundcolor=\color{white},           % choose the background color; you must add \usepackage{color} or \usepackage{xcolor}
    basicstyle=\ttfamily\bfseries,              % the size of the fonts that are used for the code
    belowskip=0.5\baselineskip,            % from: http://tex.stackexchange.com/questions/118730/avoid-empty-vert-space-after-lstlisting
    breakatwhitespace=false,                 % sets if automatic breaks should only happen at whitespace
    breaklines=true,                         % sets automatic line breaking
    captionpos=b,                            % sets the caption-position to bottom
    commentstyle=\color{commentcolor},       % comment style
    deletekeywords={...},                    % if you want to delete keywords from the given language
    escapeinside={\%*}{*)},                  % if you want to add LaTeX within your code
    extendedchars=true,                      % lets you use non-ASCII characters; for 8-bits encodings only, does not work with UTF-8
    frame=single,                            % adds a frame around the code
    %frame=none,                              % doesn't add a frame around the code
    keepspaces=true,                         % keeps spaces in text, useful for keeping indentation of code (possibly needs columns=flexible)
    %columns=fixed,                           %
    columns=flexible,                        %
    keywordstyle=\color{keywordcolor},       % keyword style
    morekeywords={%
        call,%Called method
        def,%Defined method
        inverted,%Inverted interface
        type,typedef,valueOf,                                    % Type related keywords
        method,endmethod,action,endaction,                  % Methods and actions
        Action,ActionValue,interface,endinterface,          % Interfaces
        Vector,replicate,replicateM,                        % Vector stuff
        Bit,Int,UIng,Reg,Integer,let,tagged,union,struct,   % Basic types
        TAdd,TMul,TDiv,                                     % Type operations
        rule,endrule,return,                                % Rules keywords
        pack,unpack,zeroExtend,signExtend,                  % Common bit functions
        case,matches,endcase                                % Case statements
        synthesize,True,False,Empty,*,...},                 % Etc.
    numbers=left,                            % where to put the line-numbers; possible values are (none, left, right)
    numbersep=5pt,                           % how far the line-numbers are from the code
    numberstyle=\tiny\color{numbercolor},    % the style that is used for the line-numbers
    rulecolor=\color{black},                 % if not set, the frame-color may be changed on line-breaks within not-black text (e.g. comments (green here))
    showspaces=false,                        % show spaces everywhere adding particular underscores; it overrides 'showstringspaces'
    showstringspaces=false,                  % underline spaces within strings only
    showtabs=false,                          % show tabs within strings adding particular underscores
    stepnumber=1,                            % the step between two line-numbers. If it's 1, each line will be numbered
    stringstyle=\color{stringcolor},         % string literal style
    tabsize=2,                               % sets default tabsize to 2 spaces
    title=\lstname                           % show the filename of files included with \lstinputlisting; also try caption instead of title
}

\newcommand{\mycomment}[1]{\emph{\textcolor{red}{[#1]}}}

\begin{document}

\title{RiscyOO Design Document}
\author{Sizhuo Zhang \\ szzhang@csail.mit.edu \\ MIT CSAIL}
\date{}
\maketitle

\section{Overview}


\section{Processor Core}

\subsection{Aggressive (Optimistic) Scoreboard}\label{sec:sbaggr}

The aggressive scoreboard holds an optimistic version of the presence bits of all the physical registers.
Instructions in the execution pipeline can optimistically set the presence bits in this scoreboard even when the instruction has not yet finished execution (but it may be close to finish).
The renaming stage will check the presence bits from this scoreboard for the source registers of the renaming instruction, and unset the present bit in this scoreboard for the destination register of the renaming instruction.

The interface and module presented here are slightly different from what is the source code, because the module in the source code is used for not only this aggressive scoreboard but also another conservative scoreboard (Section~\ref{sec:prf+sbcons}).

\subsubsection{Interface}
Figure~\ref{fig:aggr-sb-ifc} shows the interface of this module.
Now we explain each interface method: 
\begin{itemize}
    \item Subinterface \emph{setReady}: provides a vector of methods for instructions in execution pipelines to set the presence bits of their physical destination registers optimistically.
    \item Subinterface \emph{eagerLookup}: provides a vector of methods for instructions at the renaming stage to check the presence bits of their physical source registers.
    \item Subinterface \emph{setBusy}: provides a vector of methods for instructions at the renaming stage to unset the presence bits of their physical destination registers.
\end{itemize}

\begin{figure}
\begin{lstlisting}[caption={}]
// PhyRegs is a struct containing all the physical regs of an instruction
// RegsReady is a struct containing the presence bits for all the physical regs of an instruction
// PhyRIndex is the index of a physical reg
// SupSize is the superscalar size
interface SbLookup;
  method RegsReady get(PhyRegs r);
endinterface
interface SbSetBusy;
  method Action set(Maybe#(PhyRIndx) dst);
endinterface
interface ScoreboardAggr#(numeric type setReadyNum);
  interface Vector#(setReadyNum, Put#(PhyRIndx)) setReady;
  interface Vector#(SupSize, SbLookup) eagerLookup;
  interface Vector#(SupSize, SbSetBusy) setBusy;
endinterface
module mkScoreboardAggr(ScoreboardAggr#(setReadyNum));
  // module implementation
endmodule
\end{lstlisting}
\caption{Interface of the aggressive scoreboard}\label{fig:aggr-sb-ifc}
\end{figure}

\noindent\textbf{Conflict Matrix:}
The conflict matrix of the interface methods is:
\begin{center}
    setReady[0] $<$ $\cdots$ $<$ setReady[setReadyNum-1] $<$ eagerLookup[0] $<$ setBusy[0] $<$ eagerLookup[1] $<$ setBusy[1] $<$ $\cdots$ $<$ eagerLookup[SupSize-1] $<$ setBusy[SupSize-1].
\end{center}
We put all methods in a total order, though the conflict matrix does not need to transitive.
We order setReady $<$ eagerLookup to match the rule ordering between instruction-execution rules (which call setReady) and the renaming rule (which calls eagerLookup).
The rule ordering is because the execution rules are in later stages of the pipeline than renaming.
Thus, eagerLookup will observe the effects of all the calls to setReady in the same cycle, and this is why we call it ``eager''.
Methods eagerLookup[$i$] and setBusy[$i$] are used by the $i^{th}$ renamed instruction at the renaming stage in each cycle.
Since the $i^{th}$ instruction should see the renaming effects of all previous instructions, we order setBusy[$0\ldots i-1$] $<$ eagerLookup[$i$].

\subsubsection{Implementation}
The implementation of the module does not contain any internal rules.
It just uses a vector of EHRs to hold the presence bits, one EHR for each bit.
The interface methods access the EHRs using the appropriate port according to the conflict matrix.

\subsubsection{Source Code}
See the followings:
\begin{itemize}
    \item module \texttt{mkScoreboardAggr} in \texttt{//procs/RV64G\_OOO/ScoreboardSynth.bsv}, and
    \item module \texttt{mkRenamingScoreboard} in file \texttt{//procs/lib/Scoreboard.bsv}.
\end{itemize}
  

\section{Uncore}

\end{document}