\subsection{Speculation Tag Manager}\label{sec:spectag}

The speculation tag manager manages all the speculation tags (including both assigned tags and free tags).
Speculation tags are checked out or assigned at rename stage, and freed when the branch resolves in the ALU execution pipeline.
It should be noted that if a branch is mispredicted, then the speculation tags of itself and younger branches are all freed.
Since the module tracks all the in-flight speculation tags, it also assigns speculation bit mask to each instruction at the rename stage.

\subsubsection{Interface}

Figure~\ref{fig:spectag-ifc} shows the interface of the speculation tag manager.
Now we explain each interface method:
\begin{itemize}
    \item Method \code{currentSpecBits}: returns the speculation bit mask that should be assigned to the instruction at the rename stage.
    
    \item Method \code{nextSpecTag}: returns the speculation tag to be checked out for a new branch instruction at the rename stage.
    The guard is false if all tags have been checked out.
    
    \item Method \code{claimSpecTag}: checks out a new speculation tag for a new branch instruction at the rename stage.
    The guard is false if all tags have been checked out.
    
    \item Method \code{canClaim}: returns the guard of methods \code{nextSpecTag} and \code{claimSpecTag}.
    
    \item Subinterface \code{specUpdate}: See Section~\ref{sec:specupdate}.
\end{itemize}

\begin{figure}
\begin{lstlisting}[caption={}]
interface SpecTagManager;
  method SpecBits currentSpecBits;
  method SpecTag  nextSpecTag;
  method Action   claimSpecTag;
  method Bool     canClaim;
  interface SpeculationUpdate specUpdate;
endinterface
module mkSpecTagManager(SpecTagManager);
  // module implementation
endmodule
\end{lstlisting}
\caption{Interface of speculation tag manager}\label{fig:spectag-ifc}
\end{figure}

\noindent\textbf{Conflict Matrix:}
The conflict matrix of the interface methods is:
\begin{itemize}
    \item \{\code{currentSpecBits}, \code{nextSpecTag}, \code{canClaim}\} $<$ \code{claimSpecTag} $<$ \code{correctSpeculation}
    \item \{\code{currentSpecBits}, \code{nextSpecTag}, \code{canClaim}\} $<$ \code{incorrectSpeculation}
    \item \code{claimSpecTag} C \code{incorrectSpeculation}
\end{itemize}
We make \code{claimSpecTag} conflict with \code{incorrectSpeculation}, because the \code{claimSpecTag} must be called by a wrong-path instruction if \code{incorrectSpeculation} is called in the same cycle.

\subsubsection{Implementation}
The module uses EHR \code{current\_spec\_bits\_ehr} to keep the current speculation bit mask.
Each bit in the mask represents if the corresponding speculation tag is held of an in-flight branch in the ROB.
The module uses a vector of registers \code{dependent\_checkpoints} to track the dependencies between speculation tags.
\code{dependent\_checkpoints[t]} is the speculation bit mask that encodes all the speculation tags that is checked out after speculation tag \code{t} (including \code{t} also).
That is, if the branch with speculation tag \code{t} is mispredicted, then all speculation tags encoded in \code{dependent\_checkpoints[t]} should be freed.

All methods access the EHR or registers directly according to the conflict matrix.
We manually create a conflict between \code{claimSpecTag} and \code{incorrectSpeculation}.

\subsubsection{Source Code}
See module \code{mkSpecTagManager} in \code{//procs/lib/SpecTagManager.bsv}.
