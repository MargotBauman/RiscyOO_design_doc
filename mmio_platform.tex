\subsection{MMIO Platform}
The MMIO platform module handles all MMIO requests from the processor cores and also post timer interrupts to processor cores.
Our system has the following MMIO addresses:
\begin{itemize}
    \item Boot rom (0x1000 to 0x1FFF): The 4KB boot rom contains the very first instructions executed when the processor starts.
    \item Memory loader (0x0100\_0000 to 0x0100\_000F): The memory loader contains two MMIO registers to control the initialization of DRAM.
    \item MSIP bits of all processors (0x0200\_0000 to 0x0200\_0000 + 4 * CoreNum - 1): The MSIP bit of core $i$ is a 4B data at 0x0200\_0000 + 4 * $i$.
    \item The \code{mtimecmp} (machine-mode time compare) register of each processor core (0x0200\_4000 to 0x0200\_0000 + 8 * CoreNum - 1): The \code{mtimecmp} register of core $i$ is a 8B data at 0x0200\_4000 + 8 * $i$.
    \item The \code{mtime} (machine-mode time) register (0x0200\_BFF8 to 0x0200\_BFFF): The \code{mtime} register is a 8B data.
    \item HTIF for character input/output: HTIF contains two MMIO 8B registers \code{tohost} and \code{fromhost}:
    \begin{itemize}
        \item \code{tohost}: is like a FIFO towards x86 host software.
        Writing \code{tohost} will enqueue any non-zero write-data to FIFO.
        Reading \code{tohost} will return the oldest data in the FIFO, and will return zero if the FIFO is empty.
        \item \code{fromhost}: is like a FIFO holding any non-zero data send by the x86 host software.
        Reading \code{fromhost} returns the oldest data in the FIFO or zero if the FIFO is empty.
        Writing zero to \code{fromhost} dequeues the FIFO.
    \end{itemize}
    Their addresses are beyond \code{0x8000\_0000} and are determined when we compile the Linux image (more precisely the BBL).
    The x86 host software will tell the processor the exact addresses of these two MMIO registers at startup time.
\end{itemize}
All of the above MMIO regions except memory loader follow the addresses used in the RISC-V ISA simulator Spike (\code{//tools/riscv-isa-sim}).

\subsubsection{Interface}

\begin{figure}
\begin{lstlisting}[caption={}]
interface MMIOPlatform;
  method Action start(Addr toHost, Addr fromHost);
  method ActionValue#(Data) to_host;
  method Action from_host(Data x);
endinterface
module mkMMIOPlatform#(
  BootRomMMIO bootRom, MemLoaderMMIO memLoader,
  Vector#(CoreNum, MMIOCoreToPlatform) cores
)(MMIOPlatform);
  // implementation
endmodule
\end{lstlisting}
\caption{Interface of MMIO platform module}\label{fig:mmio-platform-ifc}
\end{figure}

Figure~\ref{fig:mmio-platform-ifc} shows the interface of the MMIO platform interface.
The module interface \code{MMIOPlatform} contains the following fields:
\begin{itemize}
    \item Method \code{start}: sets the addresses of \code{tohost} and \code{fromhost} MMIO registers.
    \item Methods \code{to\_host} and \code{from\_host}: are FIFO interfaces connected to the x86 host software.
\end{itemize}
The module takes the following arguments:
\begin{itemize}
    \item Interface \code{bootRom}: contains FIFO interfaces to access the boot rom.
    \item Interface \code{memLoader}: contains a method to access the MMIO registers in memory loader module directly.
    \item Interface \code{cores}: is a vector of subinterfaces, each of which contains FIFOs of MMIO requests and responses for a processor core.
\end{itemize}

\subsubsection{Implementation}
The module processes MMIO requests from cores in a blocking way, i.e., there is at most one in-flight MMIO request in the system..
When the module receives a MMIO request from a core, it will forward the request to the corresponding MMIO device and wait for response, or handle the request directly if the relative MMIO registers are inside this module.
In particular, this module does not hold MSIP bits inside it.
A MMIO request from core $i$ to access the MSIP bit of core $j$ will be forwarded to the MMIO handler module of core $j$.
This MMIO platform module waits for the response from core $j$ and then forward the response to core $i$.

The module also scans all the \code{mtimecmp} registers to post timer interrupts to  processor cores if needed.
The module also broadcast the value of the \code{mtime} register to each core.

\subsubsection{Future Improvement}
The blocking way of processing MMIO requests is inefficient.
In future, this module should not hold any MMIO registers, and it serves only as a router to route MMIO requests to different MMIO devices.
Each MMIO device can decide how many requests it can process in parallel.

Another issue is the unfair arbitration between the MMIO requests from different cores.
Currently, this can trigger false alerts about system deadlocking.
For example, core 0  keeps checking the memory loader to see if DRAM initialization is done or not at start up time, and it can occupy the whole bandwidth of the MMIO platform module.
Thus, other cores cannot fetch instructions from the boot rom.
If DRAM initialization takes a long time, our deadlock detection logic will output false alerts that cores other than core 0 are deadlocking.

\subsubsection{Source Code}
See module \code{mkMMIOPlatform} in \code{//procs/lib/MMIPlatform.bsv}.
