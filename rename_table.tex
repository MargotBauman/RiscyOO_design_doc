\subsection{Renaming Table}\label{sec:rt}

The renaming table manages the mapping from architectural registers to physical registers.
The rename stage queries it to get the physical registers for the source registers of instructions and rename the destination registers of the instructions.
The commit stage calls it to recycle unused physical registers.
On mis-speculation, the renaming table also needs to be recovered.

\subsubsection{Interface}

Figure~\ref{fig:rt-ifc} shows the interface of the renaming table.
Now we explain each interface:
\begin{itemize}
    \item Subinterface \code{rename}: provides a vector of methods to be called by instructions at the rename stage.
    \begin{itemize}
        \item Method \code{getRename}: returns the physical registers of the give source architectural registers according to the current mapping.
        \item Method \code{claimRename}: claims a free physical register to rename the give destination architectural register and updates the mapping.
        \item Method \code{canRename}: returns the guard of \code{claimRename}.
    \end{itemize}
    \item Subinterface \code{commit}: provides a vector of methods to be called by instructions at the commit stage.
    \begin{itemize}
        \item Method \code{commit}: recycles the physical register that is overshadowed by the renaming made by the committing instruction.
        \item Method \code{canCommit}: returns the guard of \code{commit}.
    \end{itemize}
    \item Subinterface \code{specUpdate}: manipulates speculative states (Section~\ref{sec:specupdate}).
\end{itemize}
\emph{We assume every instruction will call method \code{claimRename} at the the rename stage.}
Even if the instruction does not have a destination register, it will claim a physical register.
This simplifies the implementation of superscalar rename and commit.

\begin{figure}
\begin{lstlisting}[caption={}]
interface RTRename;
  method PhyRegs getRename(ArchRegs r);
  method Action claimRename(ArchRegs r, SpecBits sb);
  method Bool canRename;
endinterface
interface RTCommit;
  method Action commit;
  method Bool canCommit;
endinterface
interface RegRenamingTable;
  interface Vector#(SupSize, RTRename) rename;
  interface Vector#(SupSize, RTCommit) commit;
  interface SpeculationUpdate specUpdate;
endinterface
module mkRegRenamingTable(RegRenamingTable);
  // module implementation
endmodule
\end{lstlisting}
\caption{Interface of the renaming table}\label{fig:rt-ifc}
\end{figure}

\noindent\textbf{Conflict Matrix:}
The conflict matrix of the interface methods is:
\begin{itemize}
    \item \code{commit[0]} $<$ \code{commit[1]} $<$ $\cdots$ $<$ \code{commit[SupSz-1]} $<$ \code{rename[0].getRename} $<$ \\ \code{rename[0].claimRename} $<$ \code{rename[1].getRename} $<$ \code{rename[1].claimRename} $<$ $\cdots$ $<$ \code{rename[SupSize-1].getRename} $<$ \code{rename[SupSize-1].claimRename}
    \item \code{incorrectSpeculation} $<$ \code{correctSpeculation}
    \item \code{commit} $<$ \code{incorrectSpeculation}
    \item \code{incorrectSpeculation} C \code{rename.claimRename}
\end{itemize}
We order \code{rename[0].claimRename} before \code{rename[1].getRename}, because the second instruction renamed at the rename stage should see the effects made by the first instruction renamed at the rename stage in the same cycle.

\subsubsection{Implementation}
We keep the renaming mapping made by committed instructions as a table, and keep all the renamings made by in-flight instructions in ROB as a FIFO of deltas.
We did not use EHRs in this implementation.
The bypass is limited only between superscalar renamings.
We choose to use wires to capture all the method arguments, process all the actions in a canonicalization rule, explicitly bypass values from \code{claimRename} methods to \code{getRename} methods, and artificially creates method orderings.
It should be noted that \code{getRename} methods can ignore the effects of \code{commit} methods which do not change the renaming mapping.
This is a major reason not to use EHRs.

\subsubsection{Source Code}
See module \code{mkRegRenamingTable} in \code{//procs/lib/RenamingTable.bsv}.
