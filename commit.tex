\subsection{Commit Stage}

The commit stage is logically a single rule that commit instructions in order.
In implementation, it is implemented as the following mutually exclusive rules:
\begin{itemize}
    \item Rule \code{doCommitTrap\_flush}: processes a trap, i.e., an exception or an interrupt.
    It kills all instructions in the back-end, and if the trap happens after the rename stage, we increment the epoch in the epoch manager and tell the fetch pipeline to wait for redirection.
    We also record the trap information into a register to be processed in the next cycle.
    \item Rule \code{doCommitTrap\_handle}: completes the process of the trap in rule \code{doCommitTrap\_flush}.
    This rule calls method \code{trap} of the CSR register file to update CSRs and get the next PC, and redirects and fetch pipeline.
    This rule also calls a function \code{makeSystemConsistent} to initiate the necessary flush of local storage (e.g., TLBs) and the update of local copies of CSRs in other modules (e.g., the virtual-memory related CSRs in TLBs).
    The miscellaneous rules in Section~\ref{sec:misc-rules} will truly complete the flush and update and then resume the fetch pipeline.
    \item Rule \code{doCommitSystemInst}: commits a single system instruction.
    It accesses the CSR register file, redirect fetch pipeline, and call function \code{makeSystemConsistent} if necessary.
    It should be noted that we have already incremented the epoch manager for this instruction in the rename stage, so we do not need to do it again here.
    \item Rule \code{doCommitKilledLd}: processes a mis-speculative load that violates memory ordering.
    We need to re-execute this load, so we kill all instructions in the back-end, redirect the fetch pipeline to the PC of the load, and increment the epoch manager.
    \item Rule \code{notifyLSQCommit}: fires when the oldest instruction in ROB is a memory instruction that needs to be executed right at commit time (e.g., a MMIO or load-reserve or store-conditional or AMO).
    We notify the LSQ that the instruction has arrived at the commit slot and can start execution.
    \item Rule \code{doCommitNormalInst}: performs superscalar commits of normal instructions.
    If we have committed a store, we also notify LSQ that the store is committed, and thus, the store can be retired from the LSQ.
\end{itemize}
The notification to LSQ about committing/committed instructions are done using \emph{wires}.
That is, rules \code{notifyLSQCommit} and \code{doCommitNormalInst} records the the instructions in wires, and another rule reads the values of the wires and call the \code{setAtCommit} methods of LSQ.
We use wires here because of rule-scheduling issues: the ordering of the \code{setAtCommit} methods and other methods in LSQ are incompatible with the rule-scheduling conventional (i.e., reverse pipeline ordering).
The use of wires is unsatisfactory because the correctness argument of using wires is quite subtle.
The wires can raise problems if we are killing all instructions in the back-end and notifying LSQ about committed instructions in the same cycle.
This is because the \code{incorrectSpeculation} method that kills all instructions need to check the \code{committed} bits in LSQ which are set by the \code{setAtCommit} method.
Fortunately, this case can never happen in our implementation.

\subsubsection{Future Improvement}
We need to fix the wire problem mentioned above.
One way is to change the implementation of \code{setAtCommit} of LSQ.
Instead of directly setting the \code{committed} bits in LSQ in the method, we can have a FIFO in LSQ which will be enqueued by method \code{setAtCommit}, and add a new rule inside LSQ to dequeue the FIFO and set the \code{committed} bits.
The \code{incorrectSpeculation} methods should conflict with the \code{setAtCommit} method, and cannot fire if the FIFO is not empty.
In this way, the \code{setAtCommit} method of LSQ will be conflict free with other methods of LSQ, and we can directly call the method in the commit rules.

\subsubsection{Source Code}
See module \code{mkCommitStage} in \code{//procs/RV64G\_OOO/CommitStage.bsv}.
