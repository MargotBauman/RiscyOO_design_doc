\subsection{Physical Register File and Conservative Scoreboard}\label{sec:prf+sbcons}

The physical register file contains the value for each physical register, and the conservative scoreboard contains the presence bit for each physical-register value.
Instructions at the renaming stage will unset the presence bits of the destination physical registers.
Instructions that just finish execution write data into the physical register file and set the presence bits simultaneously.
Instructions in the execution pipeline read both the physical register file and the scoreboard to get the source operand and whether the operand data is present or not.
The scoreboard is called conservative because each presence bit is set only when the data is written to the corresponding physical register.

Although the physical register file and the conservative scoreboard are currently implemented as two separate modules, we describe them as one module here because they should be accessed together.

\subsubsection{Interface}
Figure~\ref{fig:prf-sb-ifc} shows the interface of the fused module of the physical register file and the conservative scoreboard.
It should be noted that the interface shown here is different from what is in the source code, which splits this module as two separate modules.
Now we explain each interface method:
\begin{itemize}
    \item Subinterface \code{write}: provides a vector of methods for instructions that just finish execution to write the data into the physcial register file and set the presence bit.
    \item Subinterface \code{read}: provides a vector of methods for instructions at register-read stage to read both the data and presence bit of the source register.
    In case the presence bit is unset, the method returns \texttt{Invalid}.
    \item Subinterface \code{setBusy}: provides a vector of methods for instructions at the renaming stage to unset the presence bits of their physical destination registers.
    this method will be called together with the one in the aggressive scoreboard.
\end{itemize}

\begin{figure}
\begin{lstlisting}[caption={}]
// PhyRIndex is the index of a physical reg
// SupSize is the superscalar size
interface RFileWr;
  method Action wr(PhyRIndx rindx, Data data);
endinterface
interface RFileRd;
  method Maybe#(Data) rd1(PhyRIndx rindx);
  method Maybe#(Data) rd2(PhyRIndx rindx);
  method Maybe#(Data) rd3(PhyRIndx rindx);
endinterface
interface SbSetBusy;
  method Action set(Maybe#(PhyRIndx) dst);
endinterface
interface RFileSbCons#(numeric type wrNum, numeric type rdNum);
  interface Vector#(wrNum, RFileWr) write;
  interface Vector#(rdNum, RFileRd) read;
  interface Vector#(SupSize, SbSetBusy) setBusy;
endinterface
module mkRFileSbCons(RFileSbCons#(wrNum, rdNum));
  // module implementation
endmodule
\end{lstlisting}
\caption{Interface of physical register file and conservative scoreboard}\label{fig:prf-sb-ifc}
\end{figure}

\noindent\textbf{Conflict Matrix:}
The conflict matrix of the interface methods is:
\begin{itemize}
    \item \code{read} CF \{\code{write}, \code{setBusy}\}
    \item \code{write[0]} $<$ \code{write[1]} $<$ $\cdots$ $<$ \code{write[wrNum-1]} $<$ \code{setBusy[0]} $<$ \code{setBusy[1]} $<$ $\cdots$ $<$ \code{setBusy[wrNum-1]}
\end{itemize}
We set all the \code{read} methods conflict free with all the \code{write} methods, because reads do not need to reflect the effects of writes immediately.
In case an old instruction is writing a register which is being read by a younger instruction as a source operand, it is ok for the read method to return \code{Invalid} even if the write method has been called.
This simply delays the execution of the younger instruction.
In our implementation, the \code{read} method will return \code{Invalid} if a \code{write} for the same physical register is being performed in the same cycle.
This can cut off the combinational path.

Forcing \code{read} methods to be ordered before \code{write} methods can cause rule-scheduling problems, because rules that read the register file are in earlier pipeline stages than rules that write register file.
That is, the method ordering will not agree with the (desired) top-level rule ordering.

We also set all the \code{read} methods conflict free with all the \code{setBusy} methods.
This is because the renaming algorithm guarantees that a physical register that is being used as a source operand of a (currently correct-path) instruction cannot be a free register for renaming destination architectural registers.
However, \emph{it is still desirable in the furture implementation to set all the \code{read} methods $<$ all the \code{setBusy} methods}, because \code{reads} happen in later pipeline stages than \code{setBusy} does.

We order all the \code{write} methods before all the \code{setBusy} methods, because \code{write} methods are called in later stages in the pipeline than \code{setBusy} methods.

\subsubsection{Implementation}
The implementation uses a vector of EHRs to store the register data, and another vector of EHRs to store the presence bits.
The \code{write} methods and \code{setBusy} methods directly access the EHRs using appropriate ports according to the conflict matrix of the interface.

The \code{read} methods do not directly access the EHRs.
We use a vector of wires to retrieve value of port 0 of each EHR, and the \code{read} methods read from these wires.
This prevents any combinational path from write to read.
This is a valid implementation because the \code{read} methods are conflict free with the \code{write} and \code{setBusy} methods, and the read methods can safely ignore the effects caused by the write or setBusy methods.

\subsubsection{Source Code}
See the followings:
\begin{itemize}
    \item module \code{mkRFileSynth} in \code{//procs/RV64G\_OOO/RFileSynth.bsv},
    \item module \code{mkRFile} in \code{//procs/lib/PhysRFile.bsv},
    \item module \code{mkScoreboardCons} in \code{//procs/RV64G\_OOO/ScoreboardSynth.bsv}, and
    \item module \code{mkRenamingScoreboard} in \code{//procs/lib/Scoreboard.bsv}.
\end{itemize}

\subsubsection{Future Improvement}
There are two improvements to make in a future implementation:
\begin{enumerate}
    \item Merge the physical register file and the conservative scoreboard into one module to ensure that their methods are called together.
    \item Order the \code{read} methods before the \code{setBusy} methods, so we don't need to rely on the high-level invariant of the renaming algorithm to ensure correctness.
\end{enumerate}
