\subsection{L1 D TLB}\label{sec:d-tlb}
The D TLB contains a fully associative array of page-table entries (PTEs).
A hit takes only one cycle.
The D TLB can process multiple requests and misses in parallel, and will send responses to the core out of order.
Requests in D TLB are speculative, so D TLB also carries speculation masks for the requests.

\subsubsection{Interface}

\begin{figure}
\begin{lstlisting}[caption={}]
typedef struct {
  Addr  addr;
  Bool  write;
} TlbReq deriving(Eq, Bits, FShow);
typedef Tuple2#(Addr, Maybe#(Exception)) TlbResp;
typedef struct {
  instT inst;
  SpecBits specBits;
} DTlbReq#(type instT) deriving(Bits, Eq, FShow);
typedef struct {
  TlbResp resp;
  instT inst;
  SpecBits specBits;
} DTlbResp#(type instT) deriving(Bits, Eq, FShow);
interface DTlb#(type instT);
  method Bool flush_done;
  method Action flush;
  method Action updateVMInfo(VMInfo vm);
  method Bool noPendingReq;
  method Action procReq(DTlbReq#(instT) req);
  method DTlbResp#(instT) procResp;
  method Action deqProcResp;
  interface DTlbToParent toParent;
  interface SpeculationUpdate specUpdate;
  interface Perf#(L1TlbPerfType) perf;
endinterface
module mkDTlb#(function TlbReq getTlbReq(instT inst))(DTlb#(instT));
  // implementation
endmodule
\end{lstlisting}
\caption{Interface of D TLB}\label{fig:d-tlb-ifc}
\end{figure}

Interface \code{DTlb} takes in a type parameter \code{instT}, which is the payload of the instruction that is requesting D TLB.
Argument \code{getTlbReq} for the D TLB module extracts the necessary information from the payload (\code{instT}) to assemble the TLB request (i.e., address and whether it is a write or not).
Now we explain the details of interface \code{DTlb}:
\begin{itemize}
    \item Method \code{flush}: triggers a flush of the TLB.
    \item Method \code{flush\_done}: returns true if the TLB has completed the flush.
    \item Method \code{updateVMInfo}: updates the local copy of some virtual-memory CSRs in the TLB.
    \item Method \code{noPendingReq}: returns true if the TLB does not have any in-flight requests.
    \item Method \code{procReq}: is called by the core to send TLB requests.
    \item Methods \code{procResp} and \code{deqProcResp}: form a FIFO-like dequeue interface.
    \code{procResp} returns a TLB response, and \code{deqProcReq} dequeues it.
    \item Subinterface \code{toParent}: contains FIFO interfaces to be connected to the parent L2 TLB.
    \item Subinterface \code{specUpdate}: manipulates speculative states (Section~\ref{sec:specupdate}).
    \item Subinterface \code{perf}: is for querying performance counters.
\end{itemize}

\noindent\textbf{Conflict Matrix:}
The conflict matrix of the module is:
\begin{itemize}
    \item \code{procResp} $<$ \code{deqProcResp} $<$ \code{procReq} $<$ \code{correctSpeculation}
    \item \code{incorrectSpeculation} C \{\code{deqProcResp}, \code{procReq}\}
    \item \code{toParent} is conflict free with others.
    \item \code{flush}, \code{flush\_done}, \code{noPendingReq}, and \code{updateVMInfo} are implementation dependent.
\end{itemize}

\subsubsection{Implementation}
The implementation has an MSHR which is a vector of EHRs to keep the in-flight requests.
THe MSHR entry will carry the speculation mask for each request.
It also contains a fully associative array containing the cached PTEs.

Method \code{procReq} allocates a MSHR entry for the new request, and searches the fully associative PTE array directly.
If the search hits, then the MSHR entry is marked as ready to respond.
Otherwise, the method marks the request as a miss and requests the parent L2 TLB.
An internal rule \code{doPRs} receives responses from L2 TLB and marks the corresponding requests as ready to respond.
The internal rule \code{doPRs} has the following ordering relations:
\begin{itemize}
    \item \code{deqProcResp} $<$ \code{doPRs} $<$ \code{procReq}
    \item \code{incorrectSpeculation} C \code{doPRs}
\end{itemize}

We implemented an optimization to avoid duplicate requests to L2 TLB.
When we are trying to issue a request to L2 TLB, if another request with the same virtual page number has already issued a request to L2 TLB, then we do not issue a duplicate one.
The \code{doPRs} rule will try to satisfy as many in-flight requests as possible with a single response from L2 TLB.

As for flushing the TLB, after the \code{flush} method is called, the D TLB will stop receiving new requests.
After all in-flight requests have been responded, the fully associative array of PTEs is cleared.
After that, D TLB requests L2 TLB to flush (via the \code{toParent} interface).
The L2 TLB will start flushing after receiving flush requests from both the I TLB and the D TLB, and will respond I and D TLBs when it finishes the flush.
After D TLB receives the flush response from the L2 TLB, the flush is done.

\subsubsection{Future Improvement}
The \code{flush} and \code{flush\_done} methods directly access some flag registers, and create some orderings between methods, which could complicate the scheduling of top-level rules of the processor core.
A better implementation should export the flush functionality as a pair of request and response FIFO interfaces which are conflict-free with each other.

\subsubsection{Source Code}
See module \code{mkDTlb} in \code{//procs/lib/DTlb.bsv}.
